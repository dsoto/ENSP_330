\documentclass[12pt, oneside]{article}
\usepackage{soto-homework}
\usepackage{siunitx}

% % page formatting
% \pagestyle{empty}
% \setlength{\parindent}{0pt}
% \setlength{\parskip}{10pt}


\begin{document}

{\bf Quiz 1 \hfill ENSP 330}


Assumptions:

Assume 130 MJ in one gallon of gasoline

Assume you drive 12,000 miles in a year in a car with 30 miles per gallon

Assume a microwave is 1000 W

Assume 1 kWh = 3.6 MJ

1 MJ = $10^6$ J, 1 GJ = $10^9$ J


\problem{}
A Kilowatt-hour (kWh) is a unit of

a) energy

b) power

\solution{energy}



\problem{}
Kilowatt (kW) is a unit of

a) energy

b) power

\solution{power}



\problem{}
How much chemical energy is contained the gasoline your car consumes in a year?

a) 52 MJ

b) 520 MJ

c) 52 GJ

\solution{
$$ 12000 miles \cdot \frac{gallons}{30 miles} \frac{130 MJ}{gallon} = 52 GJ $$
}

\vfill

\problem{}
Which is the best estimate of energy used when making popcorn?

a) $ 1000W \cdot 1 sec \cdot \frac{1 hr}{3600 sec} = 0.278 Wh $

b) $ 1000W \cdot 1 min \cdot \frac{1 hr}{60 min} = 16.7 Wh $

c) $ 1000W \cdot 1 hour  = 1 kWh $



\solution{
$$1000W \cdot 1 min$$
}





%%%%%%%%%%%%%%%%%%%%%%%%%%%%%%%%%%%%%%%%%%%%%%%%%%%%%%%%%%%%%%%%%%%
\newpage
\setcounter{problem}{0}
{\bf Quiz 2 \hfill ENSP 330}

\problem{}
Which of these two activities requires more power?

a) walking the stairs to the top of the student center

b) jumping to the top of the student center


\problem{}
In terms of the energy used, which is most accurate?

a) walking to the top takes much more energy than jumping to the top of the
building

b) walking to the top takes about the same energy as jumping to the top
of the building

c) walking to the top of the building takes much less energy than
jumping to the top


\problem{}
The Bakken formation is producing 1.1 million barrels of oil per day.
At 80 USD/barrel, which is closest to the amount of money the region is
producing each second?

a) 10 USD/sec

b) 100 USD/sec

c) 1000 USD/sec

\solution{
1.1 million barrels/day * day/86400sec * 80 USD/barrel = 1000 USD/sec
}

\vfill

\problem{}
The world uses about 500 EJ (500 Exajoules) in a year.  What size power
plant will generate that much energy if it is on 24 hours a day for an
entire year?

a) $15.9 \cdot 10^{9}$ watts (15.9 GW)

b) $15.9 \cdot 10^{12}$ watts (15.9 TW)

c) $15.9 \cdot 10^{15}$ watts (15.9 PW)

\solution{
$$ power = \frac{energy}{time} $$
$$ power = \frac{500 EJ/year}{3.14 \cdot 10^7 seconds/year} = 15.9 TW $$
}

\vfill

%%%%%%%%%%%%%%%%%%%%%%%%%%%%%%%%%%%%%%%%%%%%%%%%%%%%%%%%%%%%%%%%%%%
\newpage
\setcounter{problem}{0}
{\bf Quiz 3 \hfill ENSP 330}

% more energy conversions
% something on energy conversion with efficiency
% something on capacity factors
% SSU solar question

\problem{Solar Panel Efficiency Definition}

If I want to express the efficiency of a solar panel, what is the most
likely fraction that expresses this?

a) $\frac{\textrm{area}}{\textrm{solar radiation}}$

b) $\frac{\textrm{electrical energy}}{\textrm{solar radiation}}$

c) $\frac{\textrm{solar radiation}}{\textrm{electrical energy}}$


\problem{}

A British Thermal Unit (BTU) is a unit of

a) Energy

b) Power

\solution{
The BTU is a unit of energy.  Striking a match releases approximately 1
BTU of heat.
}


\problem{}
The amount of carbon dioxide emitted per gallon of gas depends strongly
on the mileage your car gets.

True

False

\solution{
False.  A gallon of gasoline burned in any car emits roughly the same
amount of carbon dioxide and releases roughly the same amount of heat.

}

\problem{}
On an exercise bicycle with an electrical generator a human
can produce 100 watts of power for extended periods of time.  If you pay
people minimum wage to generate power, how much does it cost per
kilowatt hour?  Assume a human can generate 100 watts and is paid the
minimum wage of 9 USD per hour.

a) 11 USD/kWh

b) 90 USD/kWh

c) 9 USD/kWh

\solution{
9 USD per hour / 0.1 kW = 90 USD per kWh
}


%%%%%%%%%%%%%%%%%%%%%%%%%%%%%%%%%%%%%%%%%%%%%%%%%%%%%%%%%%%%%%%%%%%
\newpage
\setcounter{problem}{0}
{\bf Quiz 4 \hfill ENSP 330 \hfill Fall 2014}

\problem{}
There are 1500 GW of coal plants in the world generating 9168 TWh of
electricity each year.  If the carbon intensity of electricity
production is 250 gC (grams carbon) per kWh, how much carbon is emitted each year?
Recall that GtC is one giga-ton of carbon where a ton is 1000 kg.

a) 2300 GtC per year

b) 2.3 GtC per year

c) 2.3 MtC per year

\solution{
$9168 TWh *
\frac
{\SI{1e9}{\kilo\watt\hour}}
{1 TWh}
*
250 gC /kWh
*
\frac{\SI{1}{\giga\tonne C}}{\SI{1e15}{\gram C}} = 2.3 GtC/year$

\vspace{-3.0cm}
}


% IEA key world facts coal 2012 9168 TWh electricity
% pacala and socolow supplementary material

\vspace{3.0cm}

\problem{}
Two identical coal plants are built in two different places.  According
to the second law of thermodynamics, which one is likely more efficient?

a) A plant in Saudi Arabia

b) A plant in Antarctica

\solution{

The second law tells us that the highest possible efficiency for a plant
is

$$ \eta = 1 - \frac{T_C}{T_H} $$

Since the coal plants are identical, we expect $T_H$, the temperature of
the boiler, to be the same.  $T_C$ however, the temperature of the
environment, should be lower in Antarctica and the plant will be more
efficient.

}

\problem{}
Which technology emits less carbon dioxide for each kilowatt-hour of
electricity produced?

a) A coal electricity plant

b) A natural gas electricity plant

\problem{}
Burning a gallon of gasoline emits 8.887 kilograms of CO2 emissions per
gallon.  If a car get 30 miles to the gallon, how much carbon dioxide is
emitted per mile?

a) 3.3 kg CO$_2$ per mile %30/8.87

b) 0.30 kg CO$_2$ per mile

c) 266 kg CO$_2$ %30 * 8.87

\solution{
kg CO2/mile = 8.87 kg CO2/gallon * gallons/30 miles = 0.30 kg CO2 / mile
}


%%%%%%%%%%%%%%%%%%%%%%%%%%%%%%%%%%%%%%%%%%%%%%%%%%%%%%%%%%%%%%%%%%%
\newpage
\setcounter{problem}{0}
{\bf Quiz 5 \hfill ENSP 330}
%%%%%%%%%%%%%%%%%%%%%%%%%%%%%%%%%%%%%%%%%%%%%%%%%%%%%%%%%%%%%%%%%%%
\newpage
\setcounter{problem}{0}
{\bf Quiz 6 \hfill ENSP 330}
%%%%%%%%%%%%%%%%%%%%%%%%%%%%%%%%%%%%%%%%%%%%%%%%%%%%%%%%%%%%%%%%%%%
\newpage
\setcounter{problem}{0}
{\bf Quiz 7 \hfill ENSP 330}
%%%%%%%%%%%%%%%%%%%%%%%%%%%%%%%%%%%%%%%%%%%%%%%%%%%%%%%%%%%%%%%%%%%
\newpage
\setcounter{problem}{0}
{\bf Quiz Bank \hfill ENSP 330}

\problem{}
In the X region Y amount of natural gas is being flared (burned on site)
each year because it can't
be captured economically.  What is the amount of money being lost each
second?

228 bcm of natural gas flared in SSA in 2012.

\problem{}
What are the units of the energy = mgh formula if we use kilograms, g=10
m/sec, and h in meters?

a) joules

b) BTU

c) kWh



\problem{Extra Credit}
SSU has approximately X square meters of roof area.  If we cover all the
available roof area with solar panels, approximately how much solar
power could be produced at noon on a sunny day?  Assume the panels are
20 percent efficient.

a)

b)

c)

\solution{
Area A.  Peak power = A * 1000 W/m2 * efficiency
}

\problem{}

How much money does a acre of palm oil produce per year.  Compare to the
carbon sequestration of that forest area at a carbon price of X.

\solution{
http://sustainability.tufts.edu/carbon-sequestration/

2000-4000 pounds CO2 per acre per year
}


\problem{}
Three Gorges Dam has a capacity of X GW.  If it operates at full power
for an entire year, how much energy does it create during that year?

a)

b)

c)

\solution{
X GW * 8760 hours  =
}



\problem{}
If an electric car gets 100 MPGe and the grid has an intensity of X/kWh
what is the carbon per mile?  33.7 kWh = 1 gallon of gas.


a)

b)

c)


\problem{}



\solution{
33.7 * 0.15 = 5 USD per gallon equivalent

kg CO2/mile = 0.25 kg CO2 / kWh * 33.7 kWh/gallon * gallon/100 miles=
0.089 kg CO2 / mile
}

\problem{}
If I burn 1 kg of pure carbon (a good approximation for coal), what mass
of carbon dioxide to I create?

\problem{}
At the current growth rate of solar, how long will it take to reach 1 TW
(about 10\%) of the total global energy use?


\problem{}
How many square meters of solar panels do you need to support your share
of the US energy consumption?


% SJSU ENVS 119
\problem{}
Convert 100 Joules into kWh.

% SJSU ENVS 119
\problem{}
How many btu are in 55 kWh?

% SJSU ENVS 119
\problem{}
In 2005, the US used approximately 100 quads of energy annually.
Convert this to TWh (terawatt-hours) and EJ (exa-joules).

% SJSU ENVS 119
\problem{}
A 100-watt light bulb is left on all day and night for every day for
a year. (a) How much energy does this consume? Based on PG\&E’s (the Bay
Area local utility) greenhouse gas emissions factor of 0.524 lbs
CO2/kWh, (b) how much carbon (in lbs.) is emitted over the course of a
year?

% SJSU ENVS 119
\problem{}
An average home consumes 450 kWh per month. What is the annual
energy consumption expressed in (a) tons of coal equivalent (tce)? (b)
barrels of oil equivalent (bbl)?

% SJSU ENVS 119
\problem{}
In an article in the New York Times, Google announced their power
use attributed to searches is 12.5 million Watts. Calculate the carbon
footprint of a Google search based on the actual energy consumption from
this number. (Assume 1 billion searches per day). Assume the electricity
is consumed in PG\&E territory with a 0.524 lbs CO2e/kWh emissions
factor.

% SJSU ENVS 119
\problem{}
Human beings are capable of doing approximately 100 Watts of work
per person. Diablo Canyon Nuclear Power Plant is capable of 2.35 GW of
power. (a) How many human beings (turning a crankshaft) would it take to
generate 2.35 GW of power? Assume that people can only work 8 hours a
day (but get no days off!) and that 50\% of this power is converted to
useful energy (50\% efficient). (b) If you paid these people \$8 per hour,
how much would the electricity cost in \$/kWh per month?

% SJSU ENVS 119
\problem{}
The Energy Policy Act of 2005 mandates the production of 36 billion
gallons of ethanol by 2022. (a) How much energy does 36 billions gallons
of ethanol contain (in MJ)? (b) If US drivers use the same amount of
energy to get around, how many gallons of gasoline would this displace?

% SJSU ENVS 119
\problem{}
The US imports 12 million barrels of oil (bbl) per day (hint: note
that this is a rate of energy consumption because it is energy per unit
time). (a) Convert this into power (MW or GW). (b) If the US wanted to
electrify its auto fleet using nuclear power, how many Diablo Canyon
nuclear plants would be needed (each can provide 2.35 GW of power).

% SJSU ENVS 119
\problem{}
Solar energy reaches the surface of Earth at 240 W/m$^2$ (watts/square
meter). Photovoltaic cells capture energy from the sun and make
electricity at around 15\% efficiency. Assuming there are 300 million
people consuming 10kW per person, roughly how much land is needed to
generate all this electricity from solar power in square miles (mi$2$)?
Would it fit in the Mojave Desert (25,000 mi$^2$)?


\problem{}
How many kilograms of coal contain the equivalent chemical energy of the worldwide energy use (474 EJ) per year?

% H&K chapter 3 has good questions here

\problem{}

Energy density

We want to compare the chemical energy in a gallon of gasoline to the
potential energy of lifting a car?

% put a picture here from an ipevo snap?

A gallon of gasoline contains approximately 120 MJ of chemical energy.

\problem{}
If that energy were totally converted into the gravitational potential
energy of the car, how high could you lift the car?  A Honda Civic has a
weight of about 1100 kg.


a)

b) 11 kilometers

c) 1.1 kilometers

\solution{
$$ PE = mgh = 120 kJ = 1100kg \cdot 9.8 m/sec^2 \cdot h $$
$$ h = 120 kJ / 1100 kg / 9.8 = 11.0 km $$
}


% i'm guessing gas is around 30-40 MJ/kg

% a gallon of water is about 4kg so gallon of gas is ~150 MJ?

% wikipedia has 114 kBtu * 1055 J/BTU gives about 120 MJ per gallon


\problem{Nuclear Capacity Factor}
nuclear electricity per year 2700 TWh = 2700E12 Wh = 2.7E15 Wh

nuclear capacity 370 GW

370E9 * 8760 = 3.2E15 Wh

Suggests a world-wide capacity factor of 2.7/3.2 = 84\%


\problem{}
How much energy is contained in a power bar?


\problem{}

If I burn one kilogram of coal underneath a kilogram of water and all
the heat stays in the water, how much will the temperature rise?

a)
b)
c)

\problem{}
How much energy in joules is used to blow dry your hair.  Assume a hair
dryer consumes 1000W.

a) one second * 1000W
b) 5 minutes * 1000W
c) one hour * 1000W


http://www.uh.edu/~jbutler/physical/chap25mult.html

\problem{}
Imagine that all the coal electricity in the US was created at a single
plant?  Also imagine that the coal plant is served by a single train
line.  How many coal cars must be unloaded per second to supply the
necessary amount of coal?

\solution{
}

\problem{}
What volume of calcium carbonate would it take to sequester a year's
worth of energy emissions?

\problem{}
What is the carbon per degree day for air conditioning?  Heating?

\problem{}
Stock and flow analogies for water and flow, carbon rates and stocks,
power and energy.

\problem{}
Calculate the water needs per kWh for a power plant using some
simplifying assumptions.

\problem{}
On twitter an infographic claimed a gallon of gasoline is 277K peanut
butter cups worth of energy.  Do you agree?

\solution{
Hersheys site says 105 calories each (210 per package).  One calorie is
4.2 joules one kcal is 4200 joules.  Each one is about 400 kJ or 0.4 MJ.
This is about 300 in a gallon of gas
}

\problem{}
According to the press democrat in Oct 2014, Sonoma Transit is spending
about 311 K USD on free bus travel for students and veterans.  If so
many kg of carbon dioxide are avoided by these trips, what is the
conserved cost of carbon.

\problem{}

Find the breakeven leak rate for methane and global warming.

\solution{
Ingraffea says 6-7.5 percent being leaked in paper discussed in Years of
Living Dangerously.

\problem{}
Given that the waste output of a cow can be turned into energy through
biogasification, what is the equivalent constant power output of a head
of cattle?

\problem{}

What is the per acre income from an average windfarm?

\problem{}

What is the per acre income from a solar farm?

\problem{}

What percentage of natural gas is being flared?

\problem{}

Carbonated soft drinks contain about 2.3 g of CO2 per 16oz can.  How
many cans are needed to offset your carbon emissions for today?

}

\end{document}
