\documentclass[12pt, oneside]{article}
\usepackage{graphicx}
\usepackage[nohead, margin=1.0in]{geometry}
\usepackage{pythontex}

% this uses pythontex
% compile the document using > bash compile.sh

% set up whether we are printing assignment or solution
\newif\ifsolution
\solutiontrue
\solutionfalse

% page formatting
\pagestyle{empty}
\setlength{\parindent}{0pt}
\setlength{\parskip}{10pt}

% macros for formatting
\def\true{true}
\def\false{false}

% creates true false command with indented question
\newcommand{\tf}[1]
{
\vfill
\parbox[t]{0.25\textwidth}{\bf TRUE \hspace{0.1 in} FALSE }
\parbox[t]{0.75\textwidth}{#1 (2pt)}
}

% creates grading table row macro
\newcommand{\tablerow}[2]
{\rule[-0.3cm]{0cm}{1cm}#1 & #2 & \\ \hline}

% creates heading for problems
\newcommand{\problem}[1]{{\bf Problem #1}}

% creates centered heading for sections
\newcommand{\chead}[1]
{\begin{center}\large\textbf{#1}\end{center}
\hrule
\vspace{10pt}}

% solution
\newcommand{\solution}[1]
{\ifsolution
Answer: {\it #1}
\else\fi}

% end macros



\begin{document}

% header block
\begin{center}
{\bf Midterm Exam}\\
{\bf ENSP 330}\\
{\bf Energy, Technology, and Society}\\
{\bf Daniel Soto}\\
{\bf Tuesday 29 Oct 2013}\\
\end{center}


% name and signature boxes
\makebox[1.0in][l]
{Name:}
\framebox[4.5in]{\rule{0cm}{1.5cm}}\\
\vspace{0.2cm}

\makebox[1.0in][l]
{Signature:}
\framebox[4.5in]{\rule{0cm}{1.5cm}}\\
\vspace{0.8cm}


% instructions to students
\noindent
{\bf Instructions.}
\begin{itemize}
\item Clearly show your work.  Box answers to make them clear.
\item You are allowed until 11:50 to work on this midterm.
\item In order to recieve full credit, you must show your work and
justify your answers.  The correct answer without any work will recieve
little or no credit.
\end{itemize}

\vfill

% grading and scores table
\begin{center}
\begin{tabular}{|c|c|c|}
\hline
\rule[-0.3cm]{0cm}{1cm}
Question & Points & Score \\
\hline
\tablerow{Human Power}{16}
\tablerow{Short Answer}{30}
\tablerow{True/False}{24}
\tablerow{\bf{Total}}{70}
\end{tabular}
\end{center}

\vfill


%
% midterm content
%

\newpage
\chead{Human Power}

\begin{pycode}
power_W = 200
time_min = 15

kilojoules = power_W * time_min * 60 / 1000.
\end{pycode}

The StairMaster display informs me that I am exercising at a rate of
\py{power_W} Watts.  Recall that 1 joule is equal to 1 Watt second and that 1 kcal
is equal to 4186 joules.

a) \py{power_W} watts are a measure of
\hspace{1cm}
ENERGY
\hspace{1cm}
POWER
\hspace{1cm}
(Circle one.) (4 pts)

\solution{Power}

b) After a 15-minute workout at a constant \py{power_W} W, how many joules will I
have burned? (4 pts)

\solution{
Energy equals power multiplied by time.

$$ 200W \cdot 15 min \cdot \frac{60 sec}{1 min} = \py{kilojoules} kJ $$
}

\vfill

c) Convert this amount to “nutritional calories” or kcal. (4 pts)

\solution{
$$180 kJ \cdot \frac{1000J}{kJ} \cdot \frac{1 kcal}{4186 J} = 43 kcal $$
}

\vfill

d) If I ate a breakfast of waffles with maple syrup worth 500 kcal, how
long would it take me to use up those calories on the StairMaster,
assuming its display information is accurate? (4 pts)

\solution{
Here I assume that we are 100\% efficient in the conversion of food chemical
energy to kinetic energy.  Using energy equals power multiplied by time
and the rate of energy use we determined in c:

$$ 500 kcal = \frac{43 kcal}{15 min} \cdot X min $$
$$ X = 500 kcal \cdot  \frac{15 min}{43 kcal} = 174 min $$

Which is slightly less than 3 hours.
}

\vfill



\newpage
\chead{Short Answer}

Describe the important differences and similarities between how a coal
plant and a nuclear plant generate electricity. (6 pts)

\solution{Differences: The main difference is in the source of heat.  A
coal plant uses combustion of coal while a nuclear plant uses a chain
reaction of nuclear fission to create heat.

Similarities: The main
similarity is in the use of steam turbines to generate electricity
through a steam heat engine.}

\vfill
%\vspace{1 in}

Describe the important differences between a solar photovoltaic
electricity plant and a solar thermal electricity plant. (6 pts)

\solution{Solar PV electricity creates direct current electricity
without a  thermal cycle while solar thermal electricity uses technology
Rankine cycle steam turbines to generate AC electricity.}

\vfill
%\vspace{1 in}

Describe a few advantages and disadvantages of large, centralized
electricity production. (6 pts)

\solution{Advantages include economies of scale that make larger plants
more cost effective.  Disadvantages include large investments, large
effect from individual plant failures, and
transmission costs.}

\vfill
%\vspace{1 in}

What are the main uses in the U.S. economy for coal, oil, and natural gas? (6 pts)

\solution{Coal is mostly used for electricity production, oil is mostly
used for transportation, natural gas has many uses including
electricity, heating, and transportation.}

%\vspace{1 in}
\vfill

What is the original source of our food energy? (6 pts)

\solution{The chemical energy in our food was originally radiation
energy from the sun which was converted to chemical energy by plants.
Depending on your diet, you may also consume organisms that consumed
plants.  Bonus if you mention the nuclear energy from the sun.}

\vfill


\newpage
\chead{True or False?}

\tf{A 100 MW solar photovoltaic electricity plant will likely produce
more power in a year than a 100 MW coal electricity plant.}
\solution{False}

\tf{Almost all of the energy from fossil fuels is converted to
electricity in a thermal power plant.}
\solution{False}

\tf{An energy bar contains less chemical energy than a gallon of
gasoline.}
\solution{True}

\tf{A large fraction of oil is used for electricity production in the
United States.}
\solution{False}

\tf{The BTU, the joule, and the Watt are all units of energy.}
\solution{False}

\tf{The kilowatt-hour (kWh) is a unit of power.}
\solution{False}

\tf{The second law of thermodynamics tells us that energy is conserved.}
\solution{False}

\tf{The energy intensity of an economy expresses how efficiently an
economy uses energy to create wealth per unit of GDP.}
\solution{True}

\tf{The scale of your household daily electricity energy use is about 1 MWh.}
\solution{False}

\tf{A battery converts chemical energy to electrical energy}
\solution{True}

\tf{Coal is a primary energy source.}
\solution{True}

\tf{Electricity is a primary energy source.}
\solution{False}


\end{document}

% do energy questions exist on the web?

% solar questions

% nuclear questions

% gdp questions

% fossil fuel questions

% thermo questions

% energy units questions

% estimate the capacity factor of an automobile

Convert from kWh to joules.

You have 1 kg of coal, how much hot water can you create?

