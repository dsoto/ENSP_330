\documentclass[12pt, oneside]{article}
\usepackage{soto-exam}

% set up whether we are printing assignment or solution
\solutiontrue
%\solutionfalse


\begin{document}

% header block
{\bf
ENSP 330   \hfill Energy, Technology, and Society \\
Final Exam \hfill 10 Dec 2013                      \\
}

\makenameblock

% instructions to students
\noindent
\textbf{Instructions.}
\begin{itemize}

\item For problems involving calculation, explain your approach before
starting the calculation.

\item Clearly show your work.  Box answers to make them clear.

\item You are allowed until 12:50 to work on this midterm.

\item In order to recieve full credit, you must show your work and
justify your answers.  The correct answer without any work will recieve
little or no credit.

\item You are allowed a single sheet with notes on the front and back of
the page.  You are not allowed to access the internet.

\end{itemize}

\vfill

% grading and scores table
\begin{center}
\begin{tabular}{|c|c|c|}
\hline
\rule[-0.3cm]{0cm}{1cm}
Question & Points & Score \\
\hline
\tablerow{1}{30}
\tablerow{2}{24}
\tablerow{3}{20}
\tablerow{4}{20}
\tablerow{\bf{Total}}{94}
\end{tabular}
\end{center}

\vfill

%
%
\newpage
\problem{True or False? (30 points)}
%
%

2 points each.

% basic concepts and units

\tf{A joule is a unit of power.}
\solution{False}

\tf{The kilowatt-hour (kWh) is a unit of energy.}
\solution{True.  It is a unit of energy.  This can be seen by the fact
that the unit is the product of a power (kW) and a time (hours).}

% energy conversion

\tf{Raising a quantity of water to a given height requires mechanical work
equal to the weight (mass times gravity) of the water times the elevation gain.}
\solution{True}

\tf{It takes less sunlight to create a calorie of beef than a calorie of
wheat.}
\solution{False.  The cow will eat many calories of plants to create a
calorie of muscle tissue.}

% electricity generation

\tf{A 100 MW power plant operating at a capacity factor of 50\% will
generate 1200 MWh of electricity per day on average.}
\solution{True.  A capacity factor of 50\% means that the plant will
operate at its rated power 12 hours per day.  100 MW * 12 hours = 1200
MWh per day.}


\tf{Roughly two-thirds of the fuel energy that goes into making
electricity in a traditional steam generation plant ends up as waste
heat in cooling water.}
\solution{True.  The thermal efficiency is usually about 33\% which
means the other power heats the cooling water.}

\tf{A 100 MW wind power plant will likely produce
more power in a year than a 100 MW nuclear electricity plant.}
\solution{False.  The capacity factor of wind plants is around 30\%
while a nuclear plant can be 90\%.}

% hard and soft paths

\tf{According to Amory Lovins, hard energy paths require large
investments for centralized projects}
\solution{True}

% climate questions

\tf{For the same amount of electrical energy obtained, burning natural gas produces
less carbon dioxide than coal.}
\solution{True, coal emits approximately 1 kg carbon dioxide per kWh
while natural gas is below 500 g per kWh.}

\tf{We can make very accurate predictions of future global temperatures.}
\solution{False.  We can make predictions but they have a degree of
uncertainty.}

\tf{Scientists predict that as our climate changes, we will have both
more extreme hot weather and more extreme cold weather.}
\solution{False.  We do expect more hot weather, however, we expect less
extreme cold weather.  By shifting the probability distribution, we make
hot weather more likely.}

\tf{Large scale hydropower dams have low carbon emissions and no environmental
impacts.}
\solution{False.  Hydroelectricity facilities do have low carbon
emissions but they create many impacts on habitat, migration, and
seismic activity.}

% thermo

\tf{The first law of thermodynamics tells us that energy is conserved.}
\solution{True}

% combustion

\tf{Carbon dioxide is not produced when hydrocarbons are burned.}
\solution{False.  Carbon dioxide and water are the main products of
hydrocarbon combustion.}

% transportation

\tf{A full bus is likely to emit less carbon per person per mile than a
car with a single driver.}
\solution{True.  Even though the bus uses more gas per mile, since it is
divided among many people, the per person amount of emissions is lower.}

% buildings





%% midterm
%\tf{Almost all of the energy from fossil fuels is converted to
%electricity in a thermal power plant.}
%\solution{False}
%\tf{A battery converts chemical energy to electrical energy}
%\solution{True.  Chemical reactions in the battery create the
%electricity delivered.}
%\tf{Coal is a primary energy source.}
%\solution{True.  Coal occurs naturally and can be extracted.}
%\tf{Electricity is a primary energy source.}
%\solution{False.  Electricity is produced by converting sources like
%coal or wind and is therefore a secondary energy source.}
%\tf{An energy bar contains less chemical energy than a gallon of
%gasoline.}
%\solution{True}
%\tf{A large fraction of oil is used for electricity production in the
%United States.}
%\solution{False}
%\tf{Unlike generation, the transmission and distribution of electric
%power continues to be considered a “natural monopoly.”}
%\tf{The scale of your household daily electricity energy use is about 1 MWh.}
%\solution{False.  Your average power use is likely on the order of one 1
%kW.  Multiplying this by 24 hours per day gets you to about 24 kWh.  1
%MWh is too large by about a factor of 50.}
%tf{The energy intensity of an economy expresses how efficiently an
%economy uses energy to create wealth per unit of GDP.}
%\solution{True}





% energy conversion
% energy units
% trophic levels
% thermodynamics
% what is the difference between kelvin and celsius.  why do we use
% them?
% energy prices
% energy flows, major uses
% energy gdp, energy intensity
% efficiency policy
% lifecycle cost analysis
% give an example of an initial cost and recurring cost for car,
% powerplant, lightbulb
% geothermal
% transportation quesions
% does the recent cold weather suggest anything about global warming?
% look at quiz questions
% probability distribution question


%
%
\newpage
\problem{Short Answer (24 points)}
%
%
3 points each.

\begin{enumerate}

% energy history

\item Name an energy conversion technology that existed well before
James Watt patented the steam engine in 1781.

\solution{
1. Sailing ships used the wind to propel boats.\\
2. Windmills for grain or water pumping.\\
3. Waterwheels for mechanical power.\\
4. Fire.
}

\vfill

% climate change

\item Cite reasons why it is difficult to predict how the climate will change
over the next several decades.

\solution{
1) Uncertainty in emissions\\
2) Difficulty in creating accurate computer models\\
}

\vfill

% efficiency

\item Most commonly available solar photovoltaic panels are approximately 20\%
efficient.  What two energy quantities are compared in this efficiency
ratio?

\solution{
The efficiency of a solar panel is the electrical energy output divided
by the solar radiation striking the panel as input.
}

\vfill

% primary and secondary

\item What is the primary (not secondary) energy source for most transportation?

\solution{Oil or petroleum.  Distillates (gasoline, diesel, kerosene)
used in engines are secondary energy sources.}

\vfill

%\item How do we predict the climate in the future?
%
%\item We are experiencing record low temperatures in the North Bay Area
%right now.  How does this change the theory of global warming?
%
%\item Define energy in your own words.  Give two examples of units of energy.
%
%%\item Define power in your own words.  Give two examples of units of power.
%%\solution{A watt is one joule per second.  A horsepower is about 746
%%watts.}
%

% electricity
\newpage
\item Cite three current barriers to the large scale deployment of
renewable electricity sources.

\solution{
1. High upfront costs to build new power plants.\\
2. Difficulty in transporting electricity from areas with good resource
(desert, high plains) to areas with large populations.\\
3. Lack of control of the power provided since the sun and wind cannot
be controlled.\\
4. Problems of land rights or opposition because of aesthetics.
}

\vfill

% energy supply and markets
\item Explain why natural gas prices may be higher in China than they are in
the United States.

\solution{
The key issue is that natural gas is difficult and expensive to transport.  Without
sufficient domestic supply, China will have a higher gas price.  Without
the ability to import natural gas, the price will be higher.
}

\vfill

%% comparisons

%\item Match the power with the thing most likely to consume that level
%of power.\\
%toaster \hfill
%automobile \hfill
%cell phone \hfill
%San Francisco\\
%
%
%% comparisons
%
%\item Match the amount of energy with the thing most likely to contain
%that amount of energy.\\
%candy bar \hfill
%gallon of gas \hfill
%rail car of coal
%

% electricity production

\item Suppose you wish to build a solar thermal power plant instead of PV on
the same land area because it is more efficient and less expensive. What
is the most likely reason you could build a PV but not a solar thermal
power plant there?

\solution{A solar thermal plant requires a sufficient supply of water
for steam generation and cooling.  If inexpensive water were not
available, it would likely make more sense to install a photovoltaic
plant.}

%% capacity factor
%\item Estimate the capacity factor of your automobile.
%
%\solution{The capacity factor of a power plant is the fraction of time
%that it is operating.  If I were commuting to work every day, the car
%would be used approximately 1 hour per day out of a 24 hour day.  One
%divided by 24 gives a factor of about 4\%.}
%
%\item Transportation
%
%\item Life-cycle costs
%
%% electricity
%
%% energy uses

% externalities

\vfill

\item Name some unpriced externalities associated with coal-fired
electricity production.

\solution{
1. Carbon pollution at the plant.
2. Coal ash contamination of water at the ash storage site.
3. Pollution associated with coal mining.
}

\vfill

\end{enumerate}

%\item What would be the key issue if you were deciding to install a
%solar photovoltaic plant or a solar thermal plant in the desert?
% (6 pts)
%\solution{Solar PV electricity creates direct current electricity
%without a  thermal cycle while solar thermal electricity uses technology
%Rankine cycle steam turbines to generate AC electricity.}

%
%
\newpage
\problem{Longer Answers (20 points)}
%
%
10 points each.

\begin{enumerate}

\item Articulate a strategy to verify the following claim.  A 10\% tax
on gasoline would raise \$100 billion dollars per year and stop global
warming.  Detail what information you would try to gather and how you
would use it to support or refute the claim.

\solution{
5 for quality of list of information\\
5 for quality of argument

How many gallons of gasoline are sold each year?\\
How much revenue does this gasoline sale raise?\\
Will the taxing of gasoline lower the total volume of gas sold?\\
What fraction of total carbon dioxide emissions are released by
gasoline?\\
If gasoline is taxed, will another source be substituted that will also
release carbon dioxide?\\
}

\vfill

\item Choose from either the Keystone XL Pipeline, natural gas hydraulic
fracturing, or nuclear power.  List arguments for and against these
projects that include economic and environmental concerns and state
whether you think we should pursue the project.

\solution{
Pro Keystone

increased north american oil supply\\
increased employment building pipeline\\

Con Keystone

increased use of carbon intensive fossil fuel source\\
risk of spills or leaks in pipeline\\
environmental impacts on area of tar sands production\\

Pro fracking

cheap and abundant natural gas\\

Con fracking

continued reliance on fossil fuel with volatile price\\

Pro nuclear power

low carbon baseload electricity\\

Con nuclear power

political and popular concern about safety\\
expensive to build\\
nuclear waste problem\\


}

\vfill


\end{enumerate}


%
%
\newpage
\problem{Quantitative Problems (20 points)}
%
%
 10 points each.

\begin{enumerate}

\item A fully charged Tesla Model S electric car battery holds 60 kWh of
electrical energy.  What volume of gasoline in liters would contain the
same amount of energy?  Note that 1 kWh = 3.6 MJ and one liter of gas
contains 36 MJ of chemical energy.

\solution{
3 points for correct answer\\
7 points for approach

We need to know the chemical energy of gasoline, which is included in
our given information in joules per liter.  However, the electrical
energy is in units of kilowatt-hours so we have to convert.

Gasoline has an energy density of 36 MJ per liter.

$$ 60 kWh \cdot \frac{3.6 MJ}{kWh} \frac{1 liter gas}{36 MJ} = 6
liters$$

}

\vfill

\item An automobile gets 25 miles to the gallon while carrying 4
passengers.  A bus gets 10 miles to the gallon, while carrying 20
passengers.  Calculate the gallons per person per mile for the
automobile and for the bus.


\solution{
3 points for correct answer\\
7 points for approach

$$\frac{1 gallon gas}{10 miles} \cdot \frac{1}{20 passengers} = 0.005
gallon$$

$$\frac{1 gallon gas}{25 miles} \cdot \frac{1}{4 passengers} = 0.01
gallon$$

The bus, despite having lower milage, consumes less gasoline per person
per mile.
}

\vfill

\end{enumerate}

\end{document}

% short answer
% electricity production
\item Describe the important differences and similarities between how a coal
plant and a nuclear plant generate electricity. (6 pts)

\solution{Differences: The main difference is in the source of heat.  A
coal plant uses combustion of coal while a nuclear plant uses a chain
reaction of nuclear fission to create heat.\\
Similarities: The main
similarity is in the use of steam turbines to generate electricity
through a steam heat engine.}
% do energy questions exist on the web?

 % solar questions

% nuclear questions

% gdp questions

% fossil fuel questions

% thermo questions

% energy units questions

% estimate the capacity factor of an automobile


At current prices, electricity is about four times more expensive than natural gas on a per-energy basis.
Cite two technical reasons why this difference in cost is reasonable.

(1)						(2)


Roughly half of U.S. electricity is produced from

/ 16	Fuels & Atmosphere

Write a chemical reaction (unbalanced OK) that indicates the combustion of a hydrocarbon:

    	 	         +    	        	        C>      		              +       		            +   Energy

How is biomass fuel environmentally different from
hydrocarbons, even if burning it causes local air pollution?


T	F	Joules, calories, BTUs, therms, and kilowatt-hours are all units of energy.
T	F	The 2nd Law of Thermodynamics prohibits an egg from unscrambling itself.
T	F	The 1st Law of Thermodynamics prohibits a can of soda from
spontaneously cooling itself on a hot sidewalk.
T	F	Increasing the condenser temperature increases power plant efficiency.
T	F 	A renewable resource is defined as one that has no adverse environmental impacts.
T	F	Negative feedback effects may turn average global warming into average global cooling.
T	F	Fossil fuel burning accounts for roughly two-thirds of global climate forcing.

/ 10	Renewables Estimates

You are asked to assess the potential for hydroelectric power generation  from a river. What are the two crucial pieces of information to estimate the average power that can be generated?

In 1981, Tokyo Electric Co. built an Ocean Thermal Energy Conversion
(OTEC) plant on the tropical island nation of Nauru that set a world
record for producing 120 kW of electric power, of which 30 kW went to
the 8-square-mile island’s electric grid. This type of power plant
operates a heat engine between warm surface water (about 25oC) and cold
water (about 10oC) pumped up from greater depths of the ocean. The warm
and cold water serve to alternately evaporate and condense a working
fluid such as ammonia inside a closed loop, driving a turbine as it
expands: the warm water is the heat source, and the cold the cooling
water.

What is the maximum thermodynamic efficiency of such a plant?

What do you think happened to the other 90 kW of electric power?

How much land area do you estimate would be needed to provide 10,000
MWh/yr from 10\% efficient photovoltaics in a dry, sunny climate?

If the same land area (on fertile soil) were planted with a biomass crop
to be harvested year-round and burned in a thermal power plant, the
electric energy produced from this biomass (compared to PV) would most
likely be
a. 20x less	b. 10x less 	c. about the same	d. 10x more	e. 20x more

/ 10	Copenhagen

Cite five reasons why it is not easy to predict how climate will change over the next decades as a result of greenhouse gas emissions, even though the greenhouse effect itself is well understood:

Cite five reasons why it can be difficult for representatives from different nations to negotiate language for a binding international agreement on reducing greenhouse gas emissions:

